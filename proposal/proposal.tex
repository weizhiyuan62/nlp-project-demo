% !TEX TS-program = pdflatex
\documentclass[12pt]{article}

% ---------- 基本排版 ----------
\usepackage[utf8]{inputenc}      % 允许 UTF-8 源文件
\usepackage{CJKutf8}             % 中文支持(适用于 pdflatex)
\usepackage[a4paper,margin=1in]{geometry}
\usepackage{setspace}
\usepackage{indentfirst}
\usepackage{enumitem}
\usepackage{hyperref}
\usepackage{xcolor}              % 添加颜色支持

\setlength{\parskip}{0.6em}
\setlength{\parindent}{2em}
\hypersetup{
  colorlinks=true,
  linkcolor=blue,
  urlcolor=blue
}

% ---------- 文档信息 ----------
\title{NLU\,-\,DL Midterm Proposal}
\author{魏知原\quad 2300012875}
\date{} % 留空不显示日期

\begin{document}
\begin{CJK*}{UTF8}{gbsn}

\maketitle
\vspace{-1em}
\hrule
\vspace{1.2em}

\section*{一、项目题目:《“智览”——基于大模型的智能信息聚合与分析系统》}

\section*{二、项目背景}

当前互联网信息爆炸,各类媒体渠道信息质量参差不齐,人们每天面临海量的新闻、文章和资讯,但人工浏览、筛选和分析这些信息既耗时又低效,难以快速获取高质量、全面可靠的信息摘要。根据统计,普通用户每天需要花费 1-2 小时浏览各类新闻和资讯,但其中真正有价值的信息占比不足 20\%。

传统的信息聚合方法主要依赖关键词匹配、简单的文本摘要算法(如 TF-IDF、TextRank 等),往往缺乏对信息质量的深度判断和跨源的综合分析能力。这些方法存在以下局限性:
\begin{itemize}[itemsep=0.2em]
  \item 无法理解深层语义,容易遗漏重要但表述方式特殊的信息;
  \item 缺乏对信息来源可靠性的判断能力;
  \item 生成的摘要缺乏逻辑性和可读性;
  \item 无法根据用户偏好定制个性化报告。
\end{itemize}

随着大语言模型(如 Qwen、GPT 系列、Claude 等)的发展,其强大的语义理解、信息提取和生成能力为智能化信息处理提供了新的解决方案。大模型在信息处理领域已有多个成功应用案例,如微软的 Copilot News、Google 的 AI Overview 等。

本项目旨在构建一个自动化的信息分析系统"智览",通过调用大模型 API、网页搜索引擎、数据可视化等技术,自动从网络中收集指定主题和时间范围内的信息(新闻、文章、论文等),并智能生成高质量、信息全面且可靠的分析报告。相比传统方法,大模型能够:
\begin{itemize}[itemsep=0.2em]
  \item 理解复杂的语义关系,准确判断信息的相关性和重要性;
  \item 自动进行多源信息的交叉验证,识别虚假或低质量信息,提升信息可靠性;
  \item 生成结构化、逻辑清晰、易读的分析报告,辅以数据可视化;
  \item 支持多模态内容生成(文本 + 图表 + 配图),提升报告的专业性和可读性;
  \item 根据用户配置灵活调整报告风格和内容侧重点。
\end{itemize}

\section*{三、项目方案}

本项目计划实现以下自动化流程(Pipeline):

\begin{enumerate}[align=left,label=Step\arabic*:,leftmargin=*,itemsep=0.3em,topsep=0.3em]
  \item \textbf{配置输入与解析}:
  \begin{itemize}[itemsep=0.1em]
    \item 用户通过 YAML/JSON 配置文件设定关注的主题领域(如"人工智能"、"金融市场"、"国际政治"等);
    \item 指定时间范围(今日、近三天、近一周、自定义日期区间);
    \item 选择报告风格(简明新闻风格、深度分析风格、学术刊物风格等);
    \item 设定信息源偏好(官方媒体、学术期刊、社交媒体等)。
  \end{itemize}
  
  \item \textbf{多源信息采集}:
  \begin{itemize}[itemsep=0.1em]
    \item 调用搜索引擎 API(如 Bing Search API、SerpAPI)进行通用信息检索;
    \item 对特定领域调用专业数据源 API(如 arXiv API 用于学术论文、NewsAPI 用于新闻);
    \item 获取信息的标题、来源、发布时间、摘要、URL 等元数据;
    \item 对采集到的原始数据进行去重和初步清洗。
  \end{itemize}
  
  \item \textbf{智能信息筛选与分析}:
  \begin{itemize}[itemsep=0.1em]
    \item 设计专门的 Prompt 模板,将采集到的内容分批输入大模型(如 Qwen API);
    \item 让模型根据相关性、重要性、时效性、可靠性等多个维度对信息进行评分;
    \item 过滤低质量、重复或不相关的信息;
    \item 提取每条信息的关键要点、核心观点和潜在影响;
    \item 识别信息间的关联关系(如因果关系、时间序列关系等)。
  \end{itemize}
  
  \item \textbf{数据统计与可视化}:
  \begin{itemize}[itemsep=0.1em]
    \item 使用 matplotlib/seaborn 对筛选后的信息进行统计分析;
    \item 生成热点话题词云图、时间趋势折线图、信息源分布饼图等可视化图表;
    \item 将图表保存至 assets 文件夹,命名规范化以便后续引用;
    \item 记录统计数据(如总信息量、筛选率、热点关键词等)。
  \end{itemize}
  
  \item \textbf{智能报告生成}:
  \begin{itemize}[itemsep=0.1em]
    \item 设计报告生成 Prompt,将分析结果和统计信息输入大模型;
    \item 生成结构化文本报告,包含:摘要、重点新闻解读、趋势分析、相关建议等章节;
    \item 根据配置的报告风格调整语言风格和详略程度;
    \item 将报告初稿保存为 report.txt 或 Markdown 格式。
  \end{itemize}
  
  \item \textbf{多模态配图生成}:
  \begin{itemize}[itemsep=0.1em]
    \item 调用多模态大模型 API(如 DALL-E、Midjourney API 或开源替代方案);
    \item 根据报告的主题和核心内容生成相关配图(如主题插图、概念示意图等);
    \item 对生成的图片进行质量检查和筛选;
    \item 将合格的配图保存至 assets 文件夹。
  \end{itemize}
  
  \item \textbf{LaTeX 自动排版与编译}:
  \begin{itemize}[itemsep=0.1em]
    \item 使用 Jinja2 等模板引擎将报告内容、数据图表、生成配图整合到预设的 LaTeX 模板中;
    \item 自动处理特殊字符转义、图片路径引用等细节;
    \item 调用 pdflatex 或 xelatex 编译生成格式美观的 PDF 报告;
    \item 若编译失败,解析错误信息并尝试自动修复(如调整图片大小、修正格式错误)。
  \end{itemize}
  
  \item \textbf{错误处理与日志管理}:
  \begin{itemize}[itemsep=0.1em]
    \item 在整个流程中使用 try-except 机制捕获异常;
    \item 对 API 调用失败实现自动重试机制(指数退避策略);
    \item 详细记录每个步骤的执行状态、耗时、错误信息至日志文件;
    \item 若关键步骤失败超过阈值,发送告警通知;
    \item 支持断点续传,避免因单点失败导致整个流程重新执行。
  \end{itemize}
\end{enumerate}

\section*{四、计划时间点}

\begin{itemize}[align=left,leftmargin=*,itemsep=0.2em,topsep=0.2em]
  \item \textbf{2025.11.07} \quad 提交 proposal,完成项目初步方案设计和技术调研,确定使用的 API 和工具栈。
  \item \textbf{2025.11.14} \quad 完成配置模块和信息采集模块,实现基本的搜索 API 调用功能,能够从至少两个信息源获取数据。
  \item \textbf{2025.11.21} \quad 实现信息筛选与分析模块,完成 Prompt 设计和大模型 API 集成,能够对信息进行智能评分和筛选。
  \item \textbf{2025.11.28} \quad 实现数据可视化模块和报告生成模块,完成文本报告生成,能够输出基本的分析报告。
  \item \textbf{2025.12.05} \quad 实现多模态配图生成和 LaTeX 自动排版编译功能,能够生成完整的 PDF 报告。
  \item \textbf{2025.12.12} \quad 完善错误检测、日志记录功能,进行系统测试和优化,提升系统稳定性和报告质量。
  \item \textbf{2025.12.19} \quad 完成项目报告和演示 PPT,整理项目文档和代码仓库。
\end{itemize}

\end{CJK*}
\end{document}

% This is the end if the file.
% Footer: The proposal of the projects is Finally Updated at 2025-12-13
